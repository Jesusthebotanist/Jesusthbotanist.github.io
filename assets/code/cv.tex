%%%%%%%%%%%%%%%%%%%%%%%%%%%%%%%%%%%%%%%%%
% Medium Length Professional CV
% LaTeX Template
% Version 2.0 (8/5/13)
%
% This template has been downloaded from:
% http://www.LaTeXTemplates.com
%
% Original author:
% Trey Hunner (http://www.treyhunner.com/)
%
% Important note:
% This template requires the resume.cls file to be in the same directory as the
% .tex file. The resume.cls file provides the resume style used for structuring the
% document.
%
%%%%%%%%%%%%%%%%%%%%%%%%%%%%%%%%%%%%%%%%%

%------------
%	PACKAGES AND OTHER DOCUMENT CONFIGURATIONS
%------------
\documentclass{resume} % Use the custom resume.cls style

\usepackage[left=0.75in,top=0.6in,right=0.75in,bottom=0.6in]{geometry} % Document margins
\newcommand{\tab}[1]{\hspace{.2667\textwidth}\rlap{#1}}
\newcommand{\itab}[1]{\hspace{0em}\rlap{#1}}
\name{Jes\'{u}s Mart\'{i}nez-G\'{o}mez} % Your name
% \address{B-1 \\ II , U.P. 208016} % Your address
%\address{123 Pleasant Lane \\ City, State 12345} % Your secondary addess (optional)
\address{jm2722@cornell.edu} % Your phone number and email
\newcommand\ntab[1][.25cm]{\hspace*{#1}}

\usepackage{comment} % allows me to comment out a block using \begin{comment} and \end{comment}


\begin{document}

%------------
%	EDUCATION SECTION
%------------

\begin{rSection}{Education}

\textbf{Cornell University --- Ithaca, NY} \hfill 2018 - Present
\\Ph.D Candidate,  Plant Biology Section

\textbf{University of California (UC) ---  Berkeley, CA} \hfill 2016 - 2018
\\Ph.D. Student, Integrative Biology

\textbf{University of Washington (UW) ---  Seattle, WA} \hfill 2012 - 2016 
\\B.S. Molecular, Cellular, and Developmental Biology 

\end{rSection}

%------------
%	Research Experience
%------------
\begin{rSection}{Research Experience}

\textbf{Dissertation Research: UC Berkeley/Cornell} \hfill	Fall 2016 - present \\
Evolutionary and Developmental approach to investigate the origin of branch architecture in the Amaryllis family. I implement a combination of Bayesian model based methods to study trait evolution and diversification, and developmental genetic approach to asses gene function.  
architecture in Amaryllidaceae. \\
\textbf{Advisor}: Chelsea D. Specht, PhD

\textbf{Undergraduate Research: University of Washington} \hfill	Summer 2012 - 2016\\
Studied the development mechanisms of the petaloidy in the perianth of \textit{Thalictrum};
As well as the evolution of floral scent during the transitions from insect pollinated to wind pollinated flowers.  \\
\textbf{Advisor}: Verónica S. Di Stilio, PhD

\textbf{New York University School of Medicine Summer Undergraduate Research Program (NYU SURP)} \hfill	Summer 2014 \\
Investigated the environmental influence on stem cell signaling in \textit{C. elgans}
germline. \\
\textbf{Advisor}: E. Jane Albert Hubbard, PhD

\end{rSection}
%------------
%	Academic Publication
%------------
\begin{rSection}{Academic Publications}

\begin{enumerate}[leftmargin=0cm]
\item Galimba, K. D., \textbf{Mart\'{i}nez-G\'{o}mez, J.}, \& Di Stilio, V. S. (2018). Gene Duplication and Transference of Function in the paleoAP3 Lineage of Floral Organ Identity Genes. \textit{Frontiers in Plant Science}, 9, 334. 

\item Wang, T. N., Clifford, M. R., \textbf{Mart\'{i}nez-G\'{o}mez, J.}, Johnson, J. C., Riffell, J. A., \& Di Stilio, V. S. (2018). Scent matters: differential contribution of scent to insect response in flowers with insect vs. wind pollination traits. \textit{Annals of Botany}, 123.2 (2018): 289-301.

\item Howard, C. C., \textbf{Mart\'{i}nez-G\'{o}mez, J.}, Tribble, C. Digging deeper: the evolutionary complexity of underground storage organs. \textit{In Review, American Journal of Botany  Invited Paper}.

\item textbf{Mart\'{i}nez-G\'{o}mez, J.}, Sullivan, F. A., Galimba, Cot\'{e} E., Di Stilio, V. S., Natural homeotic mutants and genetic control of floral organ identity in a ranunculid. \textit{Planned submission to New Phytologist}.

\end{enumerate}

\end{rSection}


%------------
%	Research Funding, Fellowships and Scholarships
%------------
\begin{rSection}{Research Funding, Fellowships and Scholarships} \itemsep -2pt

\underline{\textit{Graduate Fellowships} } \\
NSF Graduate Research Fellowship Program \hfill Fall 2018 \\
\ntab(3 years of Graduate Stipend + Tuition) \\
\newpage
Cornell University Diversity Fellowship \hfill Spring 2019 \\
\ntab(1 semester of Graduate Stipend + Tuition) \\
\href{http://www.hellmanfellows.org/graduate-awardees/21069/}{UC Berkeley Hellman Fellowship} \hfill	Spring 2017, 2018 \\
\ntab(2 Semester Stipend) \\	
UC Berkeley Chancellor Fellowship \hfill 2016-2018 \\
\ntab(2 years of Graduate Stipend + Tuition) \\
\\
\underline{\textit{Research and Travel Funding} }
\\
Cornell  Graduate Travel Award \hfill 2018 \\
\ntab(\$600 – Travel) \\
UC Berkeley Graduate Travel Award \hfill 2018 \\
\ntab(\$900 – Travel) \\
UC Berkeley Graduate Assembly Travel Grant \hfill	2018 \\
\ntab(\$300 Travel) \\	
American Society of Plant Taxonomy Graduate Funding	 \hfill	2018 \\
\ntab(\$1000 – Research) \\
Pacific Bulb Society Mary Sue Ittner Grant \hfill 2017 \\
\ntab(\$500- Research) \\
UC Berkeley Graduate Travel Award \hfill 2017 \\
\ntab(\$900 – Travel)	\\
UC Berkeley: Integrative Biology Travel Award \hfill 2017 \\
\ntab(\$250 – Travel)	\\
University of Washington Ronald E. McNair Research Funding \hfill 2015 \& 2016 \\
\ntab(\$2,000 - Research Funding: received twice) \\
NSF Research Experience for Undergraduate (REU) Research Funding \hfill	Summer 2015 \\
\ntab(\$3,300 - Research) \\
Frye-Hotson-Rigg Department of Biology Research Scholarship \hfill Fall 2014 \\
\ntab(\$1,200 - Research) \\
\\
\underline{\textit{Undergraduate Scholarships/Internships} } \\
UW Department of Biology Excellence in Biology Scholarship \hfill 2015 \\
\ntab(\$4,800 – UW Tuition) \\
UW Education Opportunity Program Scholarship \hfill 2013, 2014 \\
\ntab(\$1,000 – UW Tuition: received twice) \\
Washington Research Foundation Distinguished Scholar \hfill 2012-2013 \\
\ntab(\$1,600 – UW Tuition) \\
University of Washington Costco Diversity Scholar\hfill 2012 \\
\ntab(\$40,000 – UW Tuition) \\
\href{https://www.sdbonline.org/choose_development_fellows}{Society for Developmental Biology: Choose Development! Internship} \hfill Summers 2013-2015 \\
\ntab Funding for three summers (\$4000 each) to work in a developmental Biology lab  \\
University of Washington ALVA GenOM Internship \hfill Summer 2012 \\
\ntab Research experience for incoming freshmen the summer before Fall Semester. 
\end{rSection}

%------------
% Relevant Courses
%------------
%\begin{rSection}{Relevant Courses}
%\itab{\textbf{Core Courses}} \tab{}  \tab{\textbf{Other %Courses}}
%\\ \itab{Fluid Mechanics \& its applications } \tab{}  %\tab{Computational Methods in Engineering}
%\\ \itab{Thermodynamics} \tab{}  \tab{Fundamental of Computing} 
%\\ \itab{Heat Transfer \& its applications} \tab{}  \tab{Probability and Statistics} 
%\\ \itab{Mass Transfer \& its applications} \tab{} \tab{Calculus \& Linear Algebra}
%\\ \itab{Transport Phenomena (ongoing)} \tab{} \tab{Introduction to Mechanics}
% \\ \itab{Process Control (ongoing)} \tab{} \tab{Electrodynamics}
%\end{rSection}

%------------
% Research Awards/ Distinctions
%------------
\begin{rSection}{Research Awards/ Distinctions}

Hispanic Scholarship Fellow	\hfill	2017 \\
Society for Developmental Biology 75th Annual Meeting \hfill 2016\\
\ntab \textbf{*Second Place in Undergrad Poster Presentations} \\
\href{https://www.youtube.com/watch?v=eRAfphm7Eac&t=16s}{UW Cultivating Discovery Undergraduate Research Video} \hfill 2016 \\
\href{http://depts.washington.edu/uwmcnair/meet-our-scholars/2015-2016-scholars/}{UW Ronald E. McNair Research Fellow} \hfill 2014 - 2016 \\
LSAMP Annual Conference  \hfill 2014 \\
\ntab \textbf{*Best Poster Presentation}

\end{rSection}

\newpage
%------------
%	Teaching and Mentorship
%------------
\begin{rSection}{Teaching and Mentorship}

\underline{\textit{Teaching}} \\
UC Berkeley Graduate Student Instructor: PMB107 Plant Morphology \hfill	Fall 2017\\
UW Undergraduate TA: BIO355 Foundations in Cellular \& Molecular Biology	\hfill Fall 2015\\   
UW Undergraduate TA: BIOL317 Plant Classification and Identification \hfill Spring 2013

\underline{\textit{Mentorship}}\\
\textbf{Lab Mentor Jason Rose} \hfill Fall 2018 - Present\\
Assisting me in data collection for Amaryllidaceae project.

\textbf{Graduate Students Mentoring Undergraduates} \hfill Fall 2018 - Present\\
Mentors undergraduate interested in graduate school, attend monthly dinner check-ins. 

\textbf{PLANTS Botany 2017 Conference Mentor} \hfill Summer 2017 \\
Served as a mentor for rising Master student Kasey Pham at the Botany 2017 conference. 

\textbf{Summer Counselor for ALVA GenOM }\hfill Summer 2016\\
Supervised a cohort of 21 incoming freshmen as part of the GenOM 
research internship. Most students were either low income, first generation Americans or students of color. Helped foster an inclusive community for students and assisted with final research poster presentation.  

\href{https://www.washington.edu/undergradresearch/jesus-martinez-gomez/}{\textbf{Undergraduate Research Leader} } \hfill 2013-2016 \\
Promoted research opportunities to undergraduates on campus. Participated
in  panels, presented at first year interest groups and assisted students with contacting faculty for undergraduate research opportunities. 

\textbf{Laboratory Mentor for Summer Student: Mary Swadener} \hfill Summer 2015\\
Summer lab mentor for undergraduate. Guided her with execution and  
presentation of project. She won best Poster in Plant section at the  
2016 Emerging Research Conference (Washington DC).

\textbf{Leader Del Futuro Mentor} \hfill 2014-2015 \\
Paired with a freshmen undergraduate from a Hispanic background. Helped 
her navigate the University of Washington and develop habits for success.	

\textbf{University of Washington Mentor Power to Success Program}			\hfill Fall 2013 \\
Mentored a incoming freshmen undergraduate and helped her navigate the 
University classes and schedules.

\textbf{University of Washington ALVA GenOM Math and Chem Tutor} \hfill Summer 2013 \\
Biweekly afternoon tutoring to incoming freshmen in calculus and 
general chemistry. 

\end{rSection}

%------------
% Synergistic Activities
%------------
\begin{rSection}{Synergistic Activities}

\textbf{School of Integrative Plant Science Diversity Working Group} \hfill Fall 2018 \\
Group of faculyt, staff, post-doc and students aimed to address and recommend in SIPS in regards to diversity, incluusion and gender. 

\textbf{Cornell Diversity Preview Weekend Application Reviewer } \hfill Fall 2018 \\
Served as a application reviewer for three day long preview weekend for student interested in applying to graduate school in biological science.  

\textbf{Cornell Botanical Garden Judys Day - 'Plants have familites too'} \hfill Fall 2018 \\
Outreach event, ran a booth "Make a Lily" to teach families about the Lily Flower

\href{http://eflower.myspecies.info/oakspringsummerschool}{\textbf{Oak Spring eFlower Summer School}} \hfill Fall 2018 \\
Ten day intensive course on the phylogenetic comparative methods to study flower evolution. Learned methods and collected data 

\end{rSection}
%------------
%	Conference Presentation
%------------
\begin{rSection}{Conference Presentation} 
\textbf{\underline{Underline} \textendash Talks}\\
\textbf{ * \textendash Presenting Author }

\begin{enumerate}[leftmargin=0cm]

1. \underline{\textbf{Mart\'{i}nez-G\'{o}mez, J*.}}, Galimba K., Sullivan A., Cot\'{e} E., Di Stilio V.
Natural homeotic mutants and genetic control of floral organ identity in a ranunculid.
Botany 2019; July 27-31, Tucson, AZ 2019. \\
\\
2. Howard, C. C., Tribble C., \underline{\textbf{Mart\'{i}nez-G\'{o}mez, J*.}}, Males J., Sosa V., Sessa E., Specht C. D., Cellinese N. 
Ontologies as a framework to clarify Geophyte Terminology. Botany 2019; July 27-31, Tucson, AZ 2019. \\
\\
3. \textbf{Mart\'{i}nez-G\'{o}mez, J.}, Song, M*., Tribble, C.,  Freyman, W., Hohna, S., Rothfels, R., Specht, C.D.
Diversification rates across lineages: how does biological meaning differ across model-based approaches?
Botany 2019; July 27-31, Tucson, AZ 2019. \\
\\
4. \underline{Tribble, C*.,} Rothfels, R., \textbf{Mart\'{i}nez-G\'{o}mez, J.} Alzate, F., Specht, C.D. Differential gene expression in tuberous vs. non-tuberous roots of the tropical monocotyledonous geophyte Bomarea multiflora (Alstroemeriaceae).
Botany 2019; July 27-31, Tucson, AZ 2019. \\
\\
5. \underline{\textbf{Mart\'{i}nez-G\'{o}mez, J*.}},Rose, I. J.,Specht, C. D.
Incorporating prior information of developmental genetics in trait evolution with the Threshold Model: The Umbel-ivable Amaryllis Umbel as a case study. 
Evolution 2019; July 27-31, Providence,  RI2019. \\
\\
6. \underline{\textbf{Mart\'{i}nez-G\'{o}mez, J.}}, Specht C. D. Evolution of Monocot Reproductive Branches: An Evo-Devo approach to investigating the origin
of the Amaryllidaceae ‘umbel’. Ecology and Evolugionary biology Graduate Symposium; December 6-7th. Ithaca, New York 2018\\
\\
7. \underline{\textbf{Mart\'{i}nez-G\'{o}mez, J.}}, Specht C. D. Evolution of Monocot Reproductive Branches: An Evo-Devo approach to investigating the origin of the Amaryllidaceae ‘umbel.’ Monocots VI 2018; October 7-12th. Natal, Brazil 2018\\
\\
8.	\textbf{Mart\'{i}nez-G\'{o}mez, J.}, Specht C. D. Phylogenetic Comparative Method illuminates Macroevolutionary origin of the Amaryllidaceae Umbel. Botany 2018; ; July 21-15, Rochester, MN 2018. \\
\\
9.	Di Stilio V*, Hartogs S, \textbf{Mart\'{i}nez-G\'{o}mez, J.}, Tank, D. Characterizing wind pollination syndrome, its tempo and mode of evolution in Thalictrum (Ranunculaceae). Botany 2018.  July 21-15, Rochester, MN 2018.\\
\\
10.	\textbf{Mart\'{i}nez-G\'{o}mez, J.}, Specht D. C. Early Inflorescence Development in Allium: Its Umbel-ievablly. Botany 2017; June 24-28, Fort Worth, TX 2017. \\
\\
11.	\textbf{Mart\'{i}nez-G\'{o}mez, J.}, Galimba K, Di Stilio V, Galimba K. Divergence of Gene Function Following Gene Duplication and its effect on Flower Development; Society for Developmental Biology 75th Annual Meeting and International Society of Differentiation 19th International Conference; Aug 4-8;  Boston, MA, 2016. \\
\textit{*Second Place in Undergrad Poster Presentations} \\
\\
12.	Swadener M, \textbf{Mart\'{i}nez-G\'{o}mez}, J., Di Stilio V,. Uncovering the Mechanism Underlying Polyploidy in Flowering Plants using Single Copy Gene  Phylogenies; Emerging Research National Conference; Feb 25-27; Washington DC, 2017 \\
\textit{*Best Place in Poster Presentation in Environmental/Ecological Category}\\
\\
13. \underline{\textbf{Mart\'{i}nez-G\'{o}mez, J.}}, Di Stilio V, Galimba K. Evolution of UFO: Testing the Conservation of Floral Gene Regulatory Networks; 19th Annual Undergraduate Research Symposium; May 20; Seattle, WA: University of Washington; 2016\\
\\
14.	\textbf{Mart\'{i}nez-G\'{o}mez, J.}, Galimba K, Di Stilio V, Galimba K. Divergence of Gene Function Following Gene Duplication and its effect on Flower Development; 24th Annual National Ronald E. McNair Research Conference; Oct 31;  Delavan, WI: University of Wisconsin-Milkaee; 2015\\
\\
15. \underline{\textbf{Mart\'{i}nez-G\'{o}mez, J.}}, Di Stilio V, Galimba K,. Divergence of protein-protein interactions following gene duplication and its effect on flower development. 18th Annual Undergraduate Research Symposium; May 15; Seattle, WA: University of Washington; 2015. \\
\\
16.	\textbf{Mart\'{i}nez-G\'{o}mez, J.}, Pekar O, Hubbard EJA. Potential link between TGFb signaling and notch signaling in C. elegans germ line development.  Society for Developmental Biology 74th Annual Meeting; July 9-13; Snowbird, UT, 2015. p. 377. \\
\\
17.	\textbf{Mart\'{i}nez-G\'{o}mez, J.}, Pekar O, Hubbard EJA. Potential link between TGFb signaling and notch signaling in C. elegans germ line development.  The Leadership  Alliance National Symposium; July 25-27; Stamford, CT, 2014. \\
\\
18.	\textbf{Mart\'{i}nez-G\'{o}mez, J.}, Galimba K, Di Stilio V. The effect of gene duplication on protein interaction affecting flower development.  Society for Developmental Biology 73rd Annual Meeting; July 17-21; Seattle, WA: Society for Developmental Biology; 2014. \\
\\
19.	\underline {\textbf{Mart\'{i}nez-G\'{o}mez, J.} }, Di Stilio V, Galimba K. Evolution of genetic pathways affecting petal and stamen development.  17th Annual Undergraduate Research Symposium; May 16; Seattle, WA: University of Washington; 2014. \\
\\
20.	Galimba K, \textbf{Mart\'{i}nez-G\'{o}mez, J.}, Di Stilio V. Gene duplication and neo-functionalization in the APETALA3 lineage of floral organ identity genes in a non-core eudicot.  Society for Developmental Biology 73rd Annual Meeting; July 17-21; Seattle, WA: Society for Developmental Biology; 2014. \\
\\
21.	\textbf{Mart\'{i}nez-G\'{o}mez, J.}, Di Stilio V. Characterizing flower organ identity genes in a homeotic mutant.  Pacific Northwest LSAMP Conference; February 14; Portland, OR: University of Washington; 2014. \\
\textit{*Best Poster Presentation}\\
\\
22.	\textbf{Mart\'{i}nez-G\'{o}mez, J.}, Di Stilio V. Characterizing flower organ identity genes in a homeotic mutant.  16th Annual Undergraduate Research Symposium; May 17; Seattle, WA: University of Washington; 2013. \\
\\
23.	\textbf{Mart\'{i}nez-G\'{o}mez, J.}, Galimba K. Characterizing the role of the B-class gene APETALA3 in a rue-anemone mutant.  Society for the Advancement of Chicanos and Native Americans in Science (SACNAS); October; Seattle, WA, 2012.
\end{enumerate}

\end{rSection}

%------------
%	Society Membership
%------------
\begin{rSection}{Society Membership}
American Society of Plant Taxonomist\\
Society of Systematic Biology \\
Society for the Study of Evolution\\

